\documentclass[11pt,oneside]{book}

% \usepackage[margin=1.75in]{geometry}
\usepackage{graphicx}
\usepackage{amsmath}
\usepackage[colorlinks,bookmarks,bookmarksnumbered,allcolors=blue]{hyperref}
\usepackage[caption=false,justification=raggedright,singlelinecheck=false]{subfig}
\usepackage{booktabs} % better tables
\usepackage[capitalise]{cleveref}
\usepackage{floatrow}
\floatsetup[table]{capposition=top}
\usepackage{listings}
\usepackage[usenames,dvipsnames]{xcolor} 
\usepackage{algpseudocode}
\usepackage{algorithm}

\graphicspath{{figures/}}

% --- chapter author -----
\makeatletter
\newcommand{\chapterauthor}[1]{%
  {\parindent0pt\vspace*{-25pt}%
  \linespread{1.1}\large\scshape#1%
  \par\nobreak\vspace*{35pt}
  }
  \@afterheading%
}
\makeatother

% % ---- header ----
% \usepackage{fancyhdr}
% \pagestyle{fancy}
% \fancyhf{}
% \fancyhead[L]{\the\year\ BYU FLOW Lab}
% \fancyhead[R]{\thepage}

%%
%% Julia definition (c) 2014 Jubobs
%%
\lstdefinelanguage{Julia}%
{morekeywords={abstract,break,case,catch,const,continue,do,else,elseif,%
		end,export,false,for,function,immutable,import,importall,if,in,%
		macro,module,otherwise,quote,return,switch,true,try,type,typealias,%
		using,while},%
	sensitive=true,%
	alsoother={\$},%
	morecomment=[l]\#,%
	morecomment=[n]{\#=}{=\#},%
	morestring=[s]{"}{"},%
	morestring=[m]{'}{'},%
}[keywords,comments,strings]%

\lstset{%
	% language         = Julia,
	basicstyle       = \ttfamily\footnotesize,
	keywordstyle     = \bfseries\color{blue},
	stringstyle      = \color{magenta},
	commentstyle     = \color{ForestGreen},
	showstringspaces = false,
	breaklines=true,
}

% don't number sections
\setcounter{secnumdepth}{-1}

\begin{document}
	
\chapter{Quadrature}
\chapterauthor{Judd Mehr}
\label{ch:quadrature}


\section{Introduction}
\label{sec:quadratureintro}

In general, quadrature is a process by which the calculus problem of determining the integral of a curve is cast as a linear algebra problem whereby a weighted sum of intelligently chosen samples approximates the true value of the integral.

\begin{equation}\label{eqn:quadrature} \int_{a}^{b} f(x) \mathrm{d}x \approx \sum_{i=1}^n w_n f(x_n) \end{equation}

Undergraduate calculus and numerical methods courses often cover rectangle, trapezoid, and simpson's quadrature schemes, and will therefore not be covered here. We will discuss the basics of Gauss-Legendre Quadrature first, a powerful method allowing the exact approximation of polynomial functions.

\section{Basic Theory of Gauss-Legendre Quadrature}
In Gauss-Legendre quadrature, we obtain the abscissae at which we sample the curve in question using points derived from the Legendre polynomials, and weights chosen wisely to get the best possible approximation given a certain number of degrees of freedom.

\subsection{Legendre Polynomials}
Legendre polynomials are a set of polynomials forming a basis of the polynomial space up to degree $n$ where $n$ is the order of the highest order polynomial in the set. To obtain the Legendre polynomials we start with the orthogonal set

\begin{equation}  \{ 1, x, x^2, \ldots , x^n \}  \end{equation}

and through the Gram-Schmidt orthogonalization process, with respect to the inner product

\begin{equation} \langle p,q \rangle = \int^1_{-1} p(x) q(x) \mathrm{d}x \end{equation}

we obtain a new set of orthogonal polynomials that spans the same function space. We then apply constant scalings to the polynomials so that \(P_n(1) = 1\)

These polynomials (scaling included) can also be calculated using Rodrigues' formula

\begin{equation}  P_n(x) = \frac{1}{2^n n!} \frac{\mathrm{d}^n}{\mathrm{d}x^n} \left(x^2 - 1 \right) ^n  \end{equation}

Legendre polynomials have several attractive properties. Here I mention just a few that will be shown to be important momentarily.  First, Legendre polynomials of degree $n$ have exactly $n$ real roots. Second, all the roots of Legendre polynomials fall in the range $(-1,1)$.

\subsection{Gaussian Weights}

Given a polynomial $p(x)$ of degree $2n-1$, we express $p$ in terms of a polynomial division.
\begin{equation} p(x) = q(x) L_n(x) + r(x) \end{equation}

where $L_n(x)$ is the $n$th degree Legendre polynomial, and $q(x)$ is some polynomial of degree $\leq n-1$, and $r(x)$ is the remainder of the division of $q$ and $L$ and is also of degree $\leq n-1$.

Integrating, we have

\begin{equation} \int_{-1}^{1} p(x) \mathrm{d}x = \int_{-1}^{1} q(x) L_n(x) \mathrm{d}x + \int_{-1}^{1} r(x) \mathrm{d}x \end{equation}

First note that the definition of our inner product mentioned earlier has appeared, because $q$ and $L$ are, by definition, orthogonal, this first term on the right side equals zero.  We can therefore approximate this part of the integral \textit{exactly} by choosing the abscissae at which to sample the curve to be the roots of the Legendre polynomial of degree $n$, which are all real and conveniently fall between the bounds of integration.  Doing this will cause the approximation of that term to equate to zero as is its true value.

Next, we can also approximate the second term on the right side \textit{exactly} by intelligently choosing weights.  With $n$ degrees of freedom, we can approximate exactly any polynomial of order $\leq n-1$ which is the order of $r$.  We do this by applying the Vandermonde matrix in this manner

\begin{equation}
\begin{bmatrix}
1    & 1 &  \dots &1 \\
x_1      & x_2 &  \dots & x_n \\
x_1^2      & x_2^2 & \dots & x_n \\
\vdots & \vdots & \vdots & \vdots\\
x_1^{n-1}      & x_2^{n-1} & \dots & x_n^{n-1}\\
\end{bmatrix}
\begin{bmatrix}
w_1 \\
w_2 \\
w_3 \\
\vdots \\
w_n \\
\end{bmatrix}
=
\begin{bmatrix}
2  \\
0  \\
\frac{2}{3}  \\
\vdots \\
\int_{-1}^1 x^{n-1}  \\
\end{bmatrix}
\end{equation}

where $x_{n}$ are the $n$th roots of the $n$th degree Legendre polynomial, $w_n$ is the $n$th weighting value, and the right hand side are the true values of the integrals from with bounds [-1, 1] from degree zero up to $n-1$.

We can also calculate the weights using the following formula

\begin{equation} w_i = \frac{2(1-x_i^2)}{(n+1)^2 [P_{n+1}(x_i)]^2 } \end{equation}

where again $x_i$ are the roots of the $n$th degree Legendre polynomial.

Thus with the intelligent choice of both abscissae and weights, we can approximate \textit{exactly} any polynomial of degree $2n-1$ with only $n$ points. This is the power of Gaussian quadrature.

\subsection{Transforming the Bounds of Integration}

Above we have shown that the integral of any polynomial of degree $2n-1$ with bounds [-1,1] can be approximated exactly using $n$ points. With a simple transformation, we can evaluate the integral with arbitrary bounds [a,b] in the following manner

\begin{equation} \int_{a}^{b} f(x) \mathrm{d}x \approx \frac{b-a}{2} \sum_{i=1}^n w_n f(\xi_n) \end{equation}

where

\begin{equation} \xi_n = \frac{b-a}{2}x_n + \frac{a+b}{2} \end{equation}

\section{Code Implementation of Gauss-Legendre Quadrature}
After understanding the basic theory, the implementation becomes relatively simple. In the Julia language, there is a FastGaussQuadrature package that efficiently solves for the abscissae and weights for an $n$th degree approximation, making any efforts of mine to do so redundant. To find the polynomial roots, the package uses analytic expressions up to degree 5, Newton's method up to degree 60, and asymptotic expansion for degree greater than 60. Using that package, an implementation of this quadrature scheme is as follows.

\begin{lstlisting}[language=julia]
"""
gaussquadrature(fun, n; lb=-1, ub=1)

Evaluate the numeric integral of the function, fun, from the lowerbound, lb, to the upper bound, ub, using an n-th degree Gauss-Legendre quadrature rule.
"""
function gaussquadrature(fun, n; lb=-1, ub=1)
#compute abscissae and weights using FastGaussQuadrature
t, w = FastGaussQuadrature.gausslegendre(n)

#transform bounds
a = (ub-lb).*t./2.0 .+ (lb+ub)/2.0
d = (ub-lb)/2

#initialize integral
integral = 0.0

#calculate numeric integral:
#Evaluate the function (or spline object) at the abscissae and multiply by the weights; sum the values together to obtain an estimate of the integral.
for i=1:n
integral += w[i]*d*fun(a[i])
end

return integral
end
\end{lstlisting}


\section{Application to Strongly Singular Integrals}
In \hyperref[ch:boundaryelementmethod]{Boundary Element Methods (BEM)}, we often find ourselves evaluating singular integrals. Presented here is one method for doing so. Sometimes referred to as the Subtraction of Singularity Method (SST) \label{vocab:subtractionofsingularity} as described by Guiggiani and Casalini,\cite{Guiggiani1987Direct-computat} this method, in which the integral over singularities of order $1/d$ are approximated by calculating the Cauchy principal value, is specifically applicable to Isogeometric BEM (IGABEM) problems where the geometry and physics are represented in terms of basis functions.

\subsection{Cauchy Principal Values}
The definition with the Cauchy principal value denoted by $\fint$ is given as

\begin{equation} \fint_{\Gamma} = \lim_{ \varepsilon \rightarrow 0^+} \left[ \int_{s_i}^{s_p -\varepsilon} + \int_{s_p + \varepsilon}^{s_f}  \right] \end{equation}

As we can see, this allows us to, in a manner of speaking, integrate over a range $\Gamma$ from $s_i$ to $s_f$ with a singularity present at $s_p$ where $s$ are curvilinear abscissae along a curve, $\Gamma$. See figure \ref{fig:cauchypv} for a representation of the integral bounds.

\begin{figure}[htb]
	\centering
	\includegraphics[width=0.35\textwidth]{cauchypv.png}
	\caption{Example geometry over which an integral is taken in the Cauchy principal value sense, where P represents the location of a singularity.\cite{Guiggiani1987Direct-computat}}
	\label{fig:cauchypv}
\end{figure}

\subsection{Calculating Cauchy Principal Value Integrals}
To show how to compute the Cauchy principal value integral we begin with the expression

\begin{equation}
\int_{\Gamma} T(P,Q) u(s) ds
\end{equation}

where the integrand is some function on the boundary $\Gamma$. The function $T$ is strongly singular, of order $1/d$, where $d=|P-Q|$. Let us define two elements in $\Gamma$, $\Gamma_b$, ranging from $s_i$ to $s_p$, and $\Gamma_a$, ranging from $s_p$ to $s_f$ such that the singular point, $P$, is at the junction of the two elements (again, see figure \ref{fig:cauchypv}).

Let us express the function $u$ in terms of basis functions, and perform a transformation of variables on our integral, such that the bounds of integration over $\Gamma_a$ and $\Gamma_b$ will be $[-1,1]$ (a convenient practice in the implementation of boundary element methods). This gives us functions 

\begin{equation}
\int_{-1}^{1} T(P,Q(\xi)) \sum^{nb}_{c=1} \left[ N^c(\xi) \right] J_b(\xi) \mathrm{d}\xi + \int_{-1}^{1} T(P,Q(\xi)) \sum^{na}_{c=1} \left[ N^c(\xi) \right]  J_a(\xi) \mathrm{d}\xi 
\end{equation}

where $J(\xi) = \mathrm{d}s/\mathrm{d}\xi$, and $nb$ and $na$ are the number of basis functions used to define $u$ on the ranges $\Gamma_b$ and $\Gamma_a$ respectively. Only one of the basis functions on each element will be non-zero at the singular point P, so we separate out the regular integrals and compute those with standard Gauss-Legendre quadrature described in equation \ref{eqn:quadrature}. We take the singular integrals, and define $g_a(\xi)$ and $g_b(\xi)$, such that $g(\xi)=T(P,Q(\xi)) N^c(\xi)$, where $c=1$ on $\Gamma_a$ and $c=nb$ on $\Gamma_b$. We then write the Cauchy principal value expression as


\begin{equation}
\fint_{\Gamma_b + \Gamma_a} = \lim_{ \varepsilon \rightarrow 0^+} \left[ \int_{-1}^{1 -\Delta \xi_b} g_b(\xi) J_b(\xi) \mathrm{d}\xi + \int_{-1 + \Delta \xi_a}^{1} g_a(\xi) J_a(\xi) \mathrm{d}\xi \right]
\end{equation}

furthermore, let $g_a(\xi) = f(\xi)/(\xi+1)$ and $g_b(\xi) = f(\xi)/(\xi-1)$ such that we now have

\begin{equation}
\fint_{\Gamma_b + \Gamma_a} = \lim_{ \varepsilon \rightarrow 0^+} \left[ \int_{-1}^{1 -\Delta \xi_b} \frac{f_b(\xi)}{\xi-1} J_b(\xi) \mathrm{d}\xi + \int_{-1 + \Delta \xi_a}^{1} \frac{f_a(\xi)}{\xi+1} J_a(\xi) \mathrm{d}\xi \right]
\end{equation}

By expanding $\varepsilon$ in its Taylor series, we can relate $\xi$ and $\varepsilon$

\begin{equation}
\varepsilon = J_b(1) \Delta \xi_b + O(\Delta \xi_b^2) = J_a(-1) \Delta \xi_a + O(\Delta \xi_a^2)
\end{equation}

Thus $\Delta \xi_b = \varepsilon/J_b(1)$ and $\Delta \xi_a = \varepsilon/J_a(-1)$. Rewriting the bounds this way, we have 

\begin{equation}
\fint_{\Gamma_b + \Gamma_a} = \lim_{ \varepsilon \rightarrow 0^+} \left[ \int_{-1}^{1 -\varepsilon/J_b(1)} \frac{f_b(\xi)}{\xi-1} J_b(\xi) \mathrm{d}\xi + \int_{-1 + \varepsilon/J_a(-1)}^{1} \frac{f_a(\xi)}{\xi+1} J_a(\xi) \mathrm{d}\xi \right]
\end{equation}

Now let $h(\xi) = f(\xi)J(\xi)$ for brevity. 

\begin{equation}
\fint_{\Gamma_b + \Gamma_a} = \lim_{ \varepsilon \rightarrow 0^+} \left[ \int_{-1}^{1 -\varepsilon/J_b(1)} \frac{h_b(\xi)}{\xi-1} \mathrm{d}\xi + \int_{-1 + \varepsilon/J_a(-1)}^{1} \frac{h_a(\xi)}{\xi+1} \mathrm{d}\xi \right]
\end{equation}

Noting that $h(\xi)$ is a regular, continuous function over each element, $\Gamma_a$ and $\Gamma_b$, and that $h_b(1) = h_a(-1)$ we add and subtract each from the appropriate integral:

\[
\begin{aligned}
\fint_{\Gamma_b + \Gamma_a} &= 
\lim_{ \varepsilon \rightarrow 0^+} 
\biggl[ 
\int_{-1}^{1 -\varepsilon/J_b(1)} \frac{h_b(\xi)+h_b(1)-h_b(1)}{\xi-1} \mathrm{d}\xi \\
&~~~ +\int_{-1 + \varepsilon/J_a(-1)}^{1} \frac{h_a(\xi)+h_a(-1)-h_a(-1)}{\xi+1} \mathrm{d}\xi 
\biggr] \\
&= 
\lim_{ \varepsilon \rightarrow 0^+} 
\biggl[
\int_{-1}^{1 -\varepsilon/J_b(1)} \frac{h_b(\xi)-h_b(1)}{\xi-1} \mathrm{d}\xi + \int_{-1}^{1 -\varepsilon/J_b(1)} \frac{h_b(1)}{\xi-1} \mathrm{d}\xi\\
&~~~~+
\int_{-1 + \varepsilon/J_a(-1)}^{1} \frac{h_a(\xi)-h_a(-1)}{\xi+1} \mathrm{d}\xi + \int_{-1 + \varepsilon/J_a(-1)}^{1} \frac{h_a(-1)}{\xi+1} \mathrm{d}\xi
\biggr]
\end{aligned}
\]

We then evaluate the terms we added to both integrals as follows (remembering that $h_b(1) = h_a(-1)$).
\[
\begin{aligned}
& ~~~~\int_{-1}^{1 -\varepsilon/J_b(1)} \frac{h_b(1)}{\xi-1} \mathrm{d}\xi + \int_{-1 + \varepsilon/J_a(-1)}^{1} \frac{h_a(-1)}{\xi+1} \mathrm{d}\xi  \\
&=\left[ h_b(1) \ln\big|\xi-1\big|\right]_{-1}^{1 -\varepsilon/J_b(1)} + \left[ h_a(-1)\ln\big|\xi+1\big|\right]_{-1 + \varepsilon/J_a(-1)}^{1} \\
&= h_b(1)\left(\ln\big|\cancel{1}-\frac{\varepsilon}{J_b(1)} - \cancel{1}\big| - \ln\big|-1-1\big| \right) \\
&~~~~~+ h_a(-1)\left(\ln\big|1+1\big| - \ln\big|\cancel{-1}+\frac{\varepsilon}{J_a(-1)} + \cancel{1}\big| \right)
\\
&= h_b(1)\left(\cancel{\ln\big|\varepsilon\big|} - \ln\big|J_b(1) \big| - \cancel{\ln\big|2\big|} \right) + h_a(-1)\left(\cancel{\ln\big|2\big|} - \cancel{\ln\big|\varepsilon\big|} + \ln\big|J_a(-1)\big| \right)\\
& = -h_b(1) \ln \big|J_b(1)\big| +h_a(-1) \ln \big|J_a(-1)\big|
\end{aligned}
\]

which we can see are unaffected by the limit. We also note that the integrands in the limit are now regular, so removing the limit (taking $\varepsilon$ to zero) yields.

\begin{equation}
\begin{split}
\fint_{\Gamma_b + \Gamma_a} &= \int_{-1}^{1} \frac{h_b(\xi)-h_b(1)}{\xi-1} \mathrm{d}\xi \\
&~~~~~+ \int_{-1}^{1} \frac{h_a(\xi)-h_a(-1)}{\xi+1} \mathrm{d}\xi\\ 
&~~~~~~~- h_b(1) \ln \big|J_b(1)\big| + h_a(-1) \ln \big|J_a(-1)\big|
\end{split}
\end{equation}

We now have two regular integrals that can be evaluated with standard Gaussian quadrature, and two logarithmic terms to compute the Cauchy principal value. It should be noted, that although regular, the $h$ terms may still yield an undefined $0/0$ term at the bounds, so some \textit{small} perturbation should be added to keep these terms defined.

\subsection{Cauchy Principal Value Computation Example}

Let us now apply this method to a very basic example:

\begin{equation}
\int_{s_2}^{s_4} \frac{s^3+1}{s} \mathrm{d}s
\end{equation}

which has a singularity at $s=0$ (seen in figure \ref{fig:cpv}), and has the analytic Cauchy principal value:

\begin{equation}
\fint_{s_2}^{s_4} \frac{s^3+1}{s} \mathrm{d}s = \frac{s_4^3-s_2^3}{3} + \ln \bigl| \frac{s_4}{s_2} \bigr| \approx 3.69314718055994
\end{equation}


Recasting in the form discussed above we have


\begin{figure}[htb]
	\centering
	\includegraphics[width=0.75\textwidth]{cpv.pdf}
	\caption{Graphical interpretation of the desired integral for $s_2=-1$, and $s_4=2$}
	\label{fig:cpv}
\end{figure}


\begin{equation}
\int_{0}^{1} \frac{s_b(\xi)^3+1}{s_b(\xi)} \sum^{3}_{c=1} \left[ N^c(\xi) \right] J_b(\xi) \mathrm{d}\xi +\int_{-1}^{0} \frac{s_a(\xi)^3+1}{s_a(\xi)} \sum^{3}_{c=1} \left[ N^c(\xi) \right] J_a(\xi) \mathrm{d}\xi
\end{equation}

where $s$ is transformed to be in terms of $\xi$ by
\begin{equation}
\begin{aligned}
s_b(\xi) &= s_1N^1(\xi) + s_2N^2(\xi) + s_3N^3(\xi) \text{  on  } \Gamma_b = (-1,0)\\
s_a(\xi) &= s_3N^1(\xi) + s_4N^2(\xi) + s_5N^3(\xi) \text{  on  } \Gamma_a = (0,1)\\
\end{aligned}
\end{equation}

and

\begin{equation}
\begin{aligned}
J_b(\xi) &= s_1N^{1'}(\xi) + s_2N^{2'}(\xi) + s_3N^{3'}(\xi) \text{  on  } \Gamma_b = (-1,0)\\
J_a(\xi) &= s_3N^{1'}(\xi) + s_4N^{2'}(\xi) + s_5N^{3'}(\xi) \text{  on  } \Gamma_a = (0,1)\\
\end{aligned}
\end{equation}

where in this case (but in general $s_1$ and $s_5$ are permitted to move)
\begin{equation}
\begin{aligned}
s_1 &= -2\\
s_2 &= -1\\
s_3 &= 0\\
s_4 &= 2\\
s_5 &= 4\\
\end{aligned}
\end{equation}

with the chosen basis functions being the quadratic Lagrangian functions (shown in figure \ref{fig:bases})

\begin{equation}
\begin{aligned}
N^1(\xi) &= \frac{\xi(\xi-1)}{2}\\
N^2(\xi) &= (1+\xi)(1-\xi)\\
N^3(\xi) &=  \frac{\xi(\xi+1)}{2}\\
\end{aligned}
\end{equation}

with derivatives

\begin{equation}
\begin{aligned}
N^{1'}(\xi) &= \frac{2\xi-1}{2}\\
N^{2'}(\xi) &= 1-2\xi\\
N^{3'}(\xi) &=  \frac{2\xi+1}{2}\\
\end{aligned}
\end{equation}

\begin{figure}[htb]
	\centering
	\includegraphics[width=0.5\textwidth]{bases.pdf}
	\caption{Basis Functions, $N^c(\xi)$}
	\label{fig:bases}
\end{figure}

The general setup of the elements compared to the bounds of integration can be better understood through an examination of figure \ref{fig:elementdef}

\begin{figure}[htb]
	\centering
	\includegraphics[width=0.5\textwidth]{elementdef.png}
	\caption{Problem geometry definition. Note that the element bounds are \textit{not} the same as the integration bounds for this problem.\cite{Guiggiani1987Direct-computat}}
	\label{fig:elementdef}
\end{figure}

We now split out the integral into regular and singular parts over ranges on either side of the singularity at $s_3$.

\begin{equation}
\begin{split}
\int_{0}^{1} \frac{s_b(\xi)^3+1}{s_b(\xi)} \sum^{2}_{c=1} \left[ N^c(\xi) \right] J(\xi) \mathrm{d}\xi
+ \int_{-1}^{0} \frac{s_a(\xi)^3+1}{s_a(\xi)} \sum^{2}_{c=1} \left[ N^c(\xi) \right] J(\xi) \mathrm{d}\xi\\
+ \fint_{0}^{1} \frac{s(\xi)^3+1}{s(\xi)} N^3(\xi) J_b(\xi) \mathrm{d}\xi
+ \fint_{-1}^{0} \frac{s(\xi)^3+1}{s(\xi)} N^1(\xi) J_a(\xi) \mathrm{d}\xi
\end{split}
\end{equation}

Focusing now on the Cauchy principal value integrals only, we apply our methodology to obtain

\begin{equation}
\begin{split}
\int_{0}^{1+\epsilon} \frac{h_b(\xi)-h_b(1-\epsilon)}{\xi-1} \mathrm{d}\xi
+ \int_{\epsilon-1}^{0} \frac{h_a(\xi)-h_a(\epsilon-1)}{\xi+1} \mathrm{d}\xi\\
- h_b(1+\epsilon) \ln \big|J_b(1)\big| + h_a(\epsilon-1) \ln \big|J_a(-1)\big|
\end{split}
\end{equation}

with $\epsilon$ being some small perturbation ($1\times 10^{-15}$ worked well) so that the numerics work out.

Combining our now regular integrals back with the already regular parts, remembering to include $\epsilon$ in the bounds for clean numerics we have

\begin{equation}
\begin{split}
\int_{0}^{1+\epsilon} \frac{s_b(\xi)^3+1}{s_b(\xi)} \sum^{2}_{c=1} \left[ N^c(\xi) \right] J(\xi) \mathrm{d}\xi
+ \int_{\epsilon-1}^{0} \frac{s_a(\xi)^3+1}{s_a(\xi)} \sum^{2}_{c=1} \left[ N^c(\xi) \right] J(\xi) \mathrm{d}\xi\\
+ \int_{0}^{1+\epsilon} \frac{h_b(\xi)-h_b(1-\epsilon)}{\xi-1} \mathrm{d}\xi
+ \int_{\epsilon-1}^{0} \frac{h_a(\xi)-h_a(\epsilon-1)}{\xi+1} \mathrm{d}\xi\\
- h_b(1+\epsilon) \ln \big|J_b(1)\big| + h_a(\epsilon-1) \ln \big|J_a(-1)\big|
\end{split}
\end{equation}

The integrals of which we compute with the standard 3rd order Gaussian-Legendre quadrature as explained above, giving us the result $\approx 3.69314718055994$, which matches the analytic solution to 14 decimal places.


\section{Application to Weakly Singular Integrals}
When considering Gauss-Legendre quadrature with respect to weakly singular integrals, we can use a generalization of Telles' method\cite{Telles1987A-self-adaptive} described by Johnston and Elliot.\cite{Johnston2001A-generalisatio} This method uses the same Gauss points calculated before, but applies transformation that maps the interval of integration onto itself, and requires the derivative of transformation at the singular point to be zero in order to smooth (refine) the integration near the singular point.  The transformation is as follows:

\begin{equation}
\xi = \gamma_q(t) = \xi_s + \delta(\xi_s,q)(t-t_s(\xi_s,q))^q
\end{equation}

where $t$ ranges from -1 to 1, $q$ is the order of the transformation (q must be an odd integer), and $\xi_s$ is the source point. The $\delta$ and $t_2$ terms are defined as

\begin{equation}
\delta(\xi_s,q) = 
\begin{cases}
2^{1-q} & \mathrm{if}~~\xi_s = \pm 1\\
2^{-q}\left[ (1+\xi_s)^{1/q} + (1-\xi_s)^{1/q} \right]^q & \mathrm{otherwise}
\end{cases}
\end{equation}


\begin{equation}
t_s(\xi_s,q) = 
\begin{cases}
\pm 1 & \mathrm{if}~~\xi_s = \pm 1\\
\frac{ (1+\xi_s)^{1/q} - (1-\xi_s)^{1/q} }{ (1+\xi_s)^{1/q} + (1-\xi_s)^{1/q} } & \mathrm{otherwise}
\end{cases}
\end{equation}

The first derivative of the transformation is

\begin{equation}
\gamma_q'(t) = q\delta(\xi_s,q)(t-t_s(\xi_s,q))^{q-1}
\end{equation}

and is used in computing the Jacobian of transformation in the integration.




\bigskip

\bibliographystyle{aiaa} 
\bibliography{quadrature.bib}{}

 
\end{document}