\documentclass[11pt,oneside]{book}

% \usepackage[margin=1.75in]{geometry}
\usepackage{graphicx}
\usepackage{caption}
\usepackage{subcaption}
\usepackage{amsmath}
\usepackage{mathrsfs}
\usepackage{nicefrac} %package for nice fractions in exponents
\usepackage{esint} %package for cauchy principal value integral \fint
\usepackage{cancel} %package to strike out maths
\usepackage[colorlinks,bookmarks,bookmarksnumbered,allcolors=blue]{hyperref}
%\usepackage[caption=false,justification=raggedright,singlelinecheck=false]{subfig} %Note Judd swapped this for the caption/subcaption packages and updated the other .tex files so that they have the right syntax. The reason is he couldn't figure out how to use subfig to do a 2x2 sub figure like that in the airfoilparams document.
\usepackage{booktabs} % better tables
\usepackage[capitalise]{cleveref}
\usepackage{floatrow}
\floatsetup[table]{capposition=top}
\usepackage{listings}
\usepackage[usenames,dvipsnames]{xcolor} 
\usepackage{pgf,tikz,pgfplots}
\usetikzlibrary{arrows.meta}
\usepackage{algpseudocode}
\usepackage{algorithm}
\usepackage{appendix}

\graphicspath{{figures/}}

% --- chapter author -----
\makeatletter
\newcommand{\chapterauthor}[1]{%
  {\parindent0pt\vspace*{-25pt}%
  \linespread{1.1}\large\scshape#1%
  \par\nobreak\vspace*{35pt}
  }
  \@afterheading%
}
\makeatother

% % ---- header ----
% \usepackage{fancyhdr}
% \pagestyle{fancy}
% \fancyhf{}
% \fancyhead[L]{\the\year\ BYU FLOW Lab}
% \fancyhead[R]{\thepage}

%--- Bibiliography Styling---
\usepackage{etoolbox}
\patchcmd{\thebibliography}
{\chapter*}
{\section*}
{}
{}
\renewcommand{\bibname}{References}


%%
%% Julia definition (c) 2014 Jubobs
%%
\lstdefinelanguage{Julia}%
{morekeywords={abstract,break,case,catch,const,continue,do,else,elseif,%
		end,export,false,for,function,immutable,import,importall,if,in,%
		macro,module,otherwise,quote,return,switch,true,try,type,typealias,%
		using,while},%
	sensitive=true,%
	alsoother={\$},%
	morecomment=[l]\#,%
	morecomment=[n]{\#=}{=\#},%
	morestring=[s]{"}{"},%
	morestring=[m]{'}{'},%
}[keywords,comments,strings]%

\lstset{%
	% language         = Julia,
	basicstyle       = \ttfamily\footnotesize,
	keywordstyle     = \bfseries\color{blue},
	stringstyle      = \color{magenta},
	commentstyle     = \color{ForestGreen},
	showstringspaces = false,
	breaklines=true,
}

% don't number sections
%\setcounter{secnumdepth}{-1}
\renewcommand*\thesection{\arabic{section}}
\usepackage{amsthm,amssymb}
\begin{document}
	
\chapter*{Splines}
\chapterauthor{Judd Mehr}
\label{ch:splines}


\section{Introduction}
\label{sec:splinesintro}

Splines are useful in many applications including, but not limited to: Engineering, Architecture, Graphic Design, and Mathematics. Splines began as physical, rather than mathematical objects. In the ship building, and later the aircraft building, industry, as well as other architectural applications, drafters would use thin pieces of wood or metal called splines that they would bend through key points to create smooth interpolative curves.  Today, splines are used much in the same way, albeit primarily in virtual applications.  In engineering, splines are most often used for data interpolation and graphic design/representation (CAD). Simply put, splines are piecewise polynomial functions; thus they behave more or less as do polynomials, but have much greater flexibility and are therefore often preferred for accuracy in interpolation and flexibility in design. There are many kinds of splines, but here we discuss Bézier Curves, B-Splines, and NURBS. (For implementations of some of the more useful concepts described here, refer to \url{http://flow.byu.edu/Splines.jl/})

\section{Bézier Curves}
\subsection{Understanding the Parametric Nature of Bézier Curves}
Bézeir curves are parametric.  To understand this concept and begin building intuition behind general spline structure, we will jump right in and create a basic quadratic Bézier curve. We start with the parametric function $P(t)$, defined as:

\begin{equation}
\label{eq:parametricrelation}
P(t) = (1-t)P_0 + tP_1
\end{equation}

where $t$ is a parameter such that $0 \leq t \leq 1$, and $P_{(\cdot)}$ are 2D control points (note that they are 2D in this example, but are not, in general, limited to 2 dimensions). Notice that when $t=0$ we are at point $P_0$ and when $t=1$ we are at point $P_1$. 

\begin{figure}[htbp]
	\centering
	\includegraphics[width=3.0in]{para1.pdf}
	\caption{Primary Parametric Curve.}
	\label{fig:para1}
\end{figure}

Next, let's define some parametric points $Q_{(\cdot)}$ as:

\begin{align}
Q_0(t) &=  (1-t)P_0 + tP_1 \\
Q_1(t) &=  (1-t)P_1 + tP_2
 \end{align} 
 
Taking the points $Q_{(\cdot)}$ and applying the same equation \cref{eq:parametricrelation}, we can recursively create another parametric curve. Several examples are shown in figure \cref{fig:para2} as we proceed through the range of $t$. If we apply equation \cref{eq:parametricrelation} once more, the connected points create the quadratic curve, $C(t)$

\begin{align}
C(t) &= (1-t)Q_0 + tQ_1 \\
&= (1-t)^2P_0 + 2t(1-t)P_1+t^2P_2 
\end{align} 

as shown in figure \cref{fig:para3}. This recursive application of parametric polynomials is how Bézier curves are defined.
 
 \begin{figure}[htbp]
 	\centering
 	\includegraphics[width=\textwidth]{para2.pdf}
 	\caption{Recursively Defined Parametric Curves.}
 	\label{fig:para2}
 \end{figure}

 
\begin{figure}[htbp]
	\centering
	\includegraphics[width=5in]{para3.pdf}
	\caption{Recursively Defined Quadratic Bézier.}
	\label{fig:para3}
\end{figure}


\subsection{Bernstein Polynomials}
Before giving a more concise, more easily implemented defintion for Bézier Curves, let us take a minute to undertand Bernstein polynomials; which are defined as

\begin{equation} B_{i,n}(t) = {n\choose i} t^i (1-t)^{n-i} ~~~~~0\leq t \leq1 \end{equation}

where \begin{equation} {n\choose i} = \frac{n!}{i!(n-i)!} \end{equation} 

\begin{figure}[htbp]
	\centering
	\includegraphics[width=3.0in]{bernstein.pdf}
	\caption{Bernstein polynomials for $n=4$.}
	\label{fig:bernstein}
\end{figure}

From figure \cref{fig:bernstein} we can see some important properties of Berstein polynomials. We won't state why they are important just yet, but it will be good to recognize them now.

\begin{enumerate}
	\item $B_{i,n}(t) > 0$ everywhere in the range of $t$.
	\item $\sum_{i=0}^n B_{i,n}(t) = 1$ everywhere in the range of $t$.
	\item $B_{i,n}(t)$ has only one maximum in the range of $t$, at $t=i/n$.
	\item The set of  $B_{i,n}(t)$ are symmetric about $t=1/2$.
\end{enumerate}

\subsection{Bézier Curve Definition Based on Bernstein Polynomials}

Bézier curves can be expressed as follows.

\begin{equation} \textbf{C}(t) = \sum^n_{i=0}B_{i,n}(t) \textbf{P}_i~~~~~0\leq t \leq1 \end{equation}

where $B_{i,n}(t)$ are the basis (sometimes called blending or shape) functions. For Bézier curves, these basis functions are indeed Bernstein polynomials. \(\textbf{P}_i\) are geometric control points.  This formulation (mathematically equivalent to polynomial power basis form derived in like manner to the example of a quadratic curve given above) for Bézier curves is attractive due to its well behaved nature in numerical applications. Furthermore, it tends to be relatively intuitive to work with.

\section{Basis Splines (B-Splines)}

\subsection{B-Spline Basis Functions}
B-Spline construction revolves basis functions, and these basis functions are dependent on knots, or the junctions of the Bézier curves which make up the B-Spline. To define a B-Spline basis and later a B-Spline curve, we first begin by defining a knot vector, $\mathcal{T} = \{t_1, t_2  \ldots t_{m}\}$, where the knots $t_i$ are non-decreasing.  (The full importance of the knot vector will be clear later.) Using this knot vector we define $N_{i,p}(t)$, the $i$th basis function of degree $p$ (note that degree and order are not synonymous when speaking of splines; in general, order = $p+1$) using the Cox-deBoor algorithm:

\begin{equation} 
\label{eqn:basis0}
N_{i,0}(t) = 
\begin{cases} 
1 & \text{if  } t_i \leq t < t_{i+1} \\
0 & \text{otherwise}
\end{cases} 
\end{equation}

\begin{equation}
\label{eqn:basisgeneral} 
N_{i,p}(t) = \frac{t-t_i}{t_{i+p}-t_i} N_{i,p-1}(t) + \frac{t_{i+p+1}-t}{t_{i+p+1}-t_{i+1}}N_{i+1,p-1}(t) \end{equation}

\bigskip

This algorithm can be visualized as a triangular scheme:
\begin{equation}
\begin{matrix}
	N_{0,0}(t) & \rightarrow & N_{0,1}(t) & \cdots & N_{0,p-1}(t) & \rightarrow & N_{0,p}(t)\\ 
	& \nearrow &  &  &  & \nearrow & \\ 
	N_{1,0}(t) & \rightarrow & N_{1,1}(t) & \cdots & N_{1,p-1}(t) &  & \\ 
	\vdots &  & \vdots &  &  &  & \\ 
	N_{n-1,0}(t) & \rightarrow & N_{n-1,1}(t) &  &  &  & \\ 
	& \nearrow &  &  &  &  & \\ 
	N_{n,0}(t) &  &  &  &  &  & \\ 
\end{matrix}
\end{equation}

The derivative of a basis function, $N_{i,p}(t)$, is given by

\begin{equation}
\label{eqn:basisderivative}
N_{i,p}'(t) = \frac{p}{t_{i+p} - t_i} N_{i,p-1}(t) - \frac{p}{t_{i+p+1} - t_{i+1}} N_{i+1,p-1}(t)
\end{equation}

(See \cref{sec:basisderivativeproof} for the proof for this.) In general, the $k$th derivative is given by

\begin{equation}
\label{eqn:basisder1}
	N_{i,p}^{(k)}(t) = p \left( \frac{N_{i,p-1}^{(k-1)}(t)}{t_{i+p} - t_i} - \frac{N_{i+1,p-1}^{(k-1)}(t)}{t_{i+p+1} - t_{i+1}} \right)
\end{equation}

or in terms of $N_{i,p-k}(t), \ldots, N_{i+k,p-k}(t)$ rather than derivatives:

\begin{equation}
\label{eqn:basisder2}
	N_{i,p}^{(k)}(t) = \frac{p!}{(p-k)!} \sum^k_{j=0} a_{k,j} N_{i+j,p-k}(t
\end{equation}

with

\begin{align}
a_{0,0} &= 1\\
a_{k,0} &= \frac{a_{k-1,0}}{t_{i+p-k+1} - t_i}\\
a_{k,j} &= \frac{a_{k-1,j} - a_{k-1,j-1}}{t_{i+p+j-k+1} - t_{i+j} } ~~~ j=1,\ldots,k-1\\
a_{k,k} &= \frac{-a_{k-1,k-1}}{t_{i+p+1} - t_{i+k} }
\end{align}


\subsection{B-Spline Definition}
As mentioned, a B-Spline is a series of Bézier curves joined end to end with junctions called knots.  The definition of a B-Spline is

\begin{equation} \textbf{C}(t) = \sum^n_{i=0}N_{i,p}(t) \textbf{P}_i~~~~~a\leq t \leq b \end{equation}

and has derivatives

\begin{equation} \textbf{C}^{(k)}(t) = \sum^n_{i=0}N^{(k)}_{i,p}(t) \textbf{P}_i~~~~~a\leq t \leq b \end{equation}

where $\textbf{P}_i$ are control points and $N_{i,p}(t) $ are the basis functions previously defined. A $p$th degree B-Spline has $n+1$ basis functions, where $n = m-p-1$, and the number of knots, $t_i$, in the knot vector, $\mathcal{T}$, over which the spline is defined, is $m+1$ and the first and last knots are repeated $p+1$ times.
 
\begin{equation} \mathcal{T} = \{\underbrace{a,\ldots,a}_{p+1},t_{p+1},\ldots,t_{m-p-1},\underbrace{b,\ldots,b}_{p+1} \}  \end{equation}

%TODO add example of Bspline showing bases and how the knot vector affects them, as well as the associated curves.  p=2 and p=3 are probably sufficient.


\section{Non-uniform Rational B-Splines (NURBS)}
Now that we have a basic understanding of B-Splines, we can move on to NURBS, which are just a superset of B-Splines, but are therefore slightly  more useful. We will look at a few useful things we can do with splines, applied to NURBS.

\subsection{NURBS Definition}
NURBS can be thought of as weighted B-Splines, and are defined as follows:

\begin{equation}\label{eqn:nurbsdef} \textbf{C}(t) = \frac{ \displaystyle \sum^n_{i=0}N_{i,p}(t) w_i \textbf{P}_i }{ \displaystyle \sum^n_{i=0}N_{i,p}(t) w_i }~~~~~a\leq t \leq b \end{equation}  

Where again $\textbf{P}_i$ are the control points, and $\{w_i\}$ are the weights and the knot vector, $\mathcal{T}$ is defined as before except that the knots specifically need \textit{not} be uniform.

To simplify this expression, we can define rational basis functions

\begin{equation} R_{i,p}(t) = \frac{ N_{i,p}(t) w_i } { \displaystyle \sum^n_{i=0}N_{i,p}(t) w_i }\end{equation}  

and our expression becomes

\begin{equation} \textbf{C}(t) = \sum^n_{i=0} R_{i,p}(t) \textbf{P}_i  \end{equation}  

%TODO Add an example here showing the fitting of a quarter circle.

We can also describe an equivalent, but nonrational (piecewise B-Spline) curve by applying the weights to the control points through the following projection

\begin{equation}
	P^w_i = [x_iw_i,y_iw_i,z_iw_i,w_i]
\end{equation}

This gives us the nonrational curve

\begin{equation}
	\textbf{C}^w(t) = \sum^n_{i=0}N_{i,p}(t) \textbf{P}^w_i
\end{equation}

Both \(\textbf{C}(t)\) and \(\textbf{C}^w(t)\) are typically referred to as NURBS curves, despite \(\textbf{C}^w(t)\) not actually being rational.

\subsection{NURBS Derivatives}
In order to make things easier, we want to find the derivatives of NURBS curves in terms of the nonrational form. The procedure proceeds as follows.

Let 

\begin{equation}
	\textbf{C}^w(t) = [\textbf{A}(t),~w(t)]
\end{equation}

where \(\textbf{A}(t)\) is the vector of the first three elements, and \(w(t)\) is the final element, of \(\textbf{C}^w(t)\). Next we take \(\textbf{C}(t)\) and multiply by 1, and put it in terms of \(\textbf{A}(t)\) and \(w(t)\), noting that \(w(t)=\sum^n_{i=0}N_{i,p}(t) w_i \)

\begin{equation}
\label{eqn:nonrationalequiv}
	\textbf{C}(t) = \frac{ w(t) \textbf{C}(t) }{ w(t) } = \frac{ \textbf{A}(t) }{ w(t) }
\end{equation}

we can thus see that \(\textbf{A}(t)\) is also the numerator of equation \cref{eqn:nurbsdef}.

Taking the first derivative of $\textbf{C}'(t)$ via the quotient rule yields

\begin{equation}
	\textbf{C}'(t) = \frac{ w(t)\textbf{A}'(t) - w'(t) \textbf{A}(t) }{ w(t)^2 }
\end{equation}

substituting in for \(\textbf{A}(t)\) from the relation from \cref{eqn:nonrationalequiv}
\begin{equation} 
	\textbf{C}'(t) = \frac{  w(t)\textbf{A}'(t) - w'(t) \textbf{C}(t)w(t) }{ w(t)^2 }
\end{equation}

then simplifying gives us the expression for the first derivative of \(\textbf{C}(t) \):
\begin{equation} 
\textbf{C}'(t) = \frac{ \textbf{A}'(t) - w'(t) \textbf{C}(t) }{ w(t) }
\end{equation}

For higher order derivatives, we apply the generalized chain rule to \(\textbf{A}(t)\)

\begin{equation} 
\textbf{A}^{(k)}(t) = (w(t)\textbf{C}(t))^{(k)} = \displaystyle \sum^k_{i=0}\binom{k}i w^{(i)} (t) \textbf{C}^{(k-i)} (t) 
\end{equation}

taking out the first element in the summand (where $i=0$)

\begin{equation} 
\textbf{A}^{(k)}(t) = w(t)\textbf{C}^{(k)}(t) + \displaystyle \sum^k_{i=1}\binom{k}i w^{(i)} (t) \textbf{C}^{(k-i)} (t) 
\end{equation}

and then rearranging for \(\textbf{C}^{(k)}(t)\), we arrive at the expression for the $k$th derivative of the NURBS curve.

\begin{equation} 
	\textbf{C}^{(k)}(t) = \frac{ \textbf{A}^{(k)}(t) - \displaystyle \sum^k_{i=1} \binom{k}i w^{(i)}(t) \textbf{C}^{(k-i)} (t) }{ w(t) }
\end{equation}

The derivatives for \(\textbf{A}^{(k)}(t)\) and \(w^{(i)}(t)\) can be calculated by applying the $k$th derivatives of their basis functions (see \cref{eqn:basisder1} or \cref{eqn:basisder2}).

\subsection{Refinement}

%\subsubsection{Bézier Extraction}
%In order to understand the different kinds of spline refinement discussed below, we must first discuss Bézier extraction.

\subsubsection{Knot Insertion}
Knot insertion, sometimes called h-refinement, is a process by which additional knots are inserted into the knot vector, allowing for greater curve flexibility without increasing the curve order. The process of inserting a single knot begins simply by adding a knot, \(\bar{t}\), into the knot vector, \(\mathcal{T}\). If we let the new knot be between two of the existing knots such that \(\bar{t} \in [t_{k},t_{k+1})\), then we form a new knot vector defined as \(\mathcal{T}=\{\bar{t}_0=t_0,\ldots,\bar{t}_k=t_k, \bar{t}_{k+1}=\bar{t},\bar{t}_{k+2}=\bar{t}_{k+1},\ldots,\bar{t}_{m+1}=\bar{t}_m \}\). We can now express some NURBS curve as

\begin{equation}
\label{eqn:knotinsertion}
	\mathbf{C}^w(t) = \sum^{n}_{i=0} N_{i,p}(t) \mathbf{P}_i^w = \sum^{n+1}_{i=0} \bar{N}_{i,p}(t) \mathbf{Q}_i^w
\end{equation}

where \(\bar{N}_{i,p}(t)\) are the bases associated with the new knot vector, and \(\mathbf{Q}_i^w\) are the new set of control points that yield the equivalent curve. Although adding an element to the knot vector is what knot insertion is named for, the largest component of the process is finding the new control points, \(\mathbf{Q}_i^w\). It should be noted that this process in no way changes the geometric or parametric representation of the curve, but rather is a process of changing the vector space basis. This could be done by solving the system of equations presented in \cref{eqn:knotinsertion}, but because the curve has compact support, which is to say there are no more than \(p+1\) non-zero basis functions in the range \([t_{k},t_{k+1})\), the control points associated with the portions of the curve outside the range where the knot was added are unaffected. This greatly simplifies the solution for \(\mathbf{Q}_i^w\) since we only need to find $p$ new control points. This solution can be expressed as follows.

\begin{equation}
\label{eqn:knotinsertioncpcalc}
	\mathbf{Q}_i^w = \alpha_i \mathbf{P}_i^w + (1-\alpha_i)\mathbf{P}_{i-1}^w
\end{equation}

\[
\alpha_i = 
\begin{cases}
1 & i\leq k-p\\
\frac{\bar{t}-t_i}{t_{i+p} - t_i} & k-p+1 \leq i \leq k\\
0 & i \geq k+1
\end{cases}
\]

%TODO Show an example (the one in the docs is fine)

The process to add multiple knots is referred to as knot refinement.  The process is more or less identical to knot insertion. Knots can either be added one at a time, or in groups. The control points outside the range where new knots are being inserted are unchanged, and any control points associated with sections of the curve where the new knots are being inserted are solved for with the same \cref{eqn:knotinsertioncpcalc} as the single knot case. It should also be noted that multiples of the same knot can be added without issue.


\subsubsection{Degree Elevation}
Degree elevation, sometimes called p-refinement, is a process by which a p-degree spline is represented exactly in a higher order space. The process of degree elevation is very similar to that of knot refinement. We simply take a $p$-degree NURBS curve and embed it into a higher dimensional space.

\begin{equation}
\label{eqn:degreeelevation}
\mathbf{C}_p^w(t) = \mathbf{C}_{p+1}^w(t) = \sum^{\hat{n}}_{i=0} N_{i,p+1}(t) \mathbf{Q}_i^w
\end{equation}

The method of degree elevation is to determine the knot vector and control points that define the identical curve in higher dimensional space.  Obtaining the knot vector is quite simple. In order for the elevated curve to be the same, we need it to have knots in the same locations, with the same continuity as the original knot vector.  The orginal knot vector has the form

\begin{equation}
	\mathcal{T} = \{t_0,\ldots,t_m\} = \{\underbrace{a,\ldots,a}_{p+1},\underbrace{t_1,\ldots,t_m}_{m_1},\ldots,\underbrace{t_s,\ldots,t_s}_{m_s},\underbrace{b,\ldots,b}_{p+1} \}
\end{equation}

where \(m_1,\ldots,m_s\) are the individual knot multiplicities. To obtain the knot vector for the higher dimensioned curve, we simply add as many multiples of the original knots as degrees we are elevating. If we are elevating the curve by degree 1, for example, the new knot vector becomes

\begin{equation}
\hat{\mathcal{T}} = \{t_0,\ldots,t_{\hat{m}}\} = \{\underbrace{a,\ldots,a}_{p+2},\underbrace{t_1,\ldots,t_m}_{m_1+1},\ldots,\underbrace{t_s,\ldots,t_s}_{m_s+1},\underbrace{b,\ldots,b}_{p+2} \}
\end{equation}

where \(\hat{m} = m+s+2\).  Having added knots, we now solve for the new set of basis functions using our elevated knot vector, the elevated degree \(p\), and \(\hat{n}\) where \(\hat{n} = n+s+1\).  After we have the new knots and bases, we finally solve for the control points that define the identical curve, very much like our knot insertion process where we have a system of equations with which we can solve for the control points.

\begin{equation}
\sum^{\hat{n}}_{i=0} N_{i,p+1}(t) \mathbf{Q}_i^w = \sum^{n}_{i=0} N_{i,p}(t) \mathbf{P}_i^w
\end{equation}

There are more efficient algorithms than solving this system to finding these control points. Like the efficient algorithms for knot refinement, they involve Bézier extraction, or degree elevating sections of the curve one at a time.

%TODO Show an example (the one in the docs is fine)



%\section{Tensor Product Surfaces}

\bibliographystyle{aiaa} 
\bibliography{splines.bib}{}


\begin{appendices}
\renewcommand{\thesection}{Appendix \Alph{section}.}

\section{Proof by Induction for Basis Functions Derivatives}
\label{sec:basisderivativeproof}
The proof of \cref{eqn:basisderivative} comes from section 2.3 in reference \cite{Piegl1997The-NURBS-Book}. 

\renewcommand{\qedsymbol}{\rule{0.7em}{0.7em}}

\begin{proof}
Let \(p=1\). By \cref{eqn:basis0} \(N_{i,p-1}(t)\) and \(N_{i+1,p-1}(t)\) are either 0 or 1. Thus by \cref{eqn:basisgeneral}, for \(p=1\), \(N_{i,p}^{'}(t)\) is either 

\[ \frac{1}{ t_{i+1} - t_{i} } ~~\text{or}~~ \frac{-1}{ t_{i+2} - t_{i+1} } \]

Assume now that \cref{eqn:basisderivative} is true for \(p-1,~p>1\). Applying the product rule, we differentiate \cref{eqn:basisgeneral} to obtain

\begin{equation}
\begin{aligned}
N_{i,p}'(t) &= \frac{1}{t_{i+p} - t_i} N_{i,p-1}(t) + \frac{t-t_i}{t_{i+p}-t_i} N_{i,p-1}^{'}(t)\\
			&+ \frac{-1}{t_{i+p+1} - t_{i+1}} N_{i+1,p-1}(t) + \frac{t_{i+p+1}-t}{t_{i+p+1}-t_{i+1}}N_{i+1,p-1}^{'}(t)
\end{aligned}
\end{equation}

Now substitute in \cref{eqn:basisderivative} for \(N_{i,p-1}^{'}(t)\) and \(N_{i+1,p-1}^{'}(t)\)

\begin{equation}
\begin{split}
N_{i,p}'(t) =& \frac{1}{t_{i+p} - t_i} N_{i,p-1}(t) + \frac{-1}{t_{i+p+1} - t_{i+1}} N_{i+1,p-1}(t) \\ 
& + \frac{t-t_i}{t_{i+p}-t_i} \left( \frac{p-1}{t_{i+p-1} - t_i} N_{i,p-2}(t) - \frac{p-1}{t_{i+p} - t_{i+1}} N_{i+1,p-2}(t) \right) \\
&+ \frac{t_{i+p+1}-t}{t_{i+p+1}-t_{i+1}} \left( \frac{p-1}{t_{i+p} - t_{i+1}} N_{i+1,p-2}(t) - \frac{p-1}{t_{i+p+1} - t_{i+2}} N_{i+2,p-2}(t) \right)
\end{split}\raisetag{3\baselineskip}
\end{equation}

Gather like terms:

\begin{equation}
\begin{split}
\label{eqn:gatherterms}
N_{i,p}'(t) =& \frac{1}{t_{i+p} - t_i} N_{i,p-1}(t)\\
& + \frac{-1}{t_{i+p+1} - t_{i+1}} N_{i+1,p-1}(t) \\ 
& + \frac{p-1}{t_{i+p-1} - t_i} ~\frac{t-t_i}{t_{i+p}-t_i} N_{i,p-2}(t)\\
& + \frac{p-1}{t_{i+p} - t_{i+1}} \left( \frac{t_{i+p+1}-t}{t_{i+p+1}-t_{i+1}} - \frac{t-t_i}{t_{i+p}-t_i}\right)  N_{i+1,p-2}\\
& - \frac{p-1}{t_{i+p+1} - t_{i+2}}~\frac{t_{i+p+1}-t}{t_{i+p+1}-t_{i+1}} N_{i+2,p-2}(t)
\end{split}\raisetag{3\baselineskip}
\end{equation}

Add and subtract 1 from the terms in parenthesis, allowing us to put things in a convenient form shortly.

\begin{equation}
\begin{split}
\label{eqn:plusminus1}
\frac{t_{i+p+1}-t}{t_{i+p+1}-t_{i+1}} - \frac{t-t_i}{t_{i+p}-t_i} =& -1 + \frac{t_{i+p+1}-t}{t_{i+p+1}-t_{i+1}} +1 - \frac{t-t_i}{t_{i+p}-t_i}\\
	=& -\frac{t_{i+p+1} - t_{i+1}}{t_{i+p+1} - t_{i+1}} + \frac{t_{i+p+1}-t}{t_{i+p+1}-t_{i+1}}\\
	&+ \frac{t_{i+p} - t_{i}}{t_{i+p} - t_{i}} - \frac{t-t_i}{t_{i+p}-t_i}\\
	=& \frac{t_{i+p} - t}{t_{i+p} - t_{i}} - \frac{t - t_{i+1}}{t_{i+p+1} - t_{i+1}}
\end{split}\raisetag{3\baselineskip}
\end{equation}

Apply \cref{eqn:plusminus1} and rearrange \cref{eqn:gatherterms}

\begin{equation}
\begin{split}
N_{i,p}'(t) =& \frac{1}{t_{i+p} - t_i} N_{i,p-1}(t)\\
& + \frac{-1}{t_{i+p+1} - t_{i+1}} N_{i+1,p-1}(t) \\ 
& + \frac{p-1}{t_{i+p} - t_i} \left( \frac{t-t_i}{t_{i+p-1}-t_i} N_{i,p-2}(t) + \frac{t_{i+p}-t}{t_{i+p}-t_{i+1}} N_{i+1,p-2}(t) \right)\\
& - \frac{p-1}{t_{i+p+1} - t_{i+1}}\left( \frac{t-t_{i+1}}{t_{i+p}-t_{i+1}} N_{i+1,p-2}(t) + \frac{t_{i+p+1}-t}{t_{i+p+1}-t_{i+2}} N_{i+2,p-2}(t) \right)
\end{split}\raisetag{3\baselineskip}
\end{equation}

We can see that the terms in parentheses now are precisely the relation defined in \cref{eqn:basisgeneral}. Substituting in the left hand side of that equation, we are left with

\begin{equation}
\begin{split}
N_{i,p}'(t) =& \frac{1}{t_{i+p} - t_i} N_{i,p-1}(t)\\
& + \frac{-1}{t_{i+p+1} - t_{i+1}} N_{i+1,p-1}(t) \\ 
& + \frac{p-1}{t_{i+p} - t_i} N_{i,p-1}(t) \\
& - \frac{p-1}{t_{i+p+1} - t_{i+1}} N_{i+1,p-1}(t)
\end{split}\raisetag{3\baselineskip}
\end{equation}

Simplifying, we obtain \cref{eqn:basisderivative}

\[
N_{i,p}'(t) = \frac{p}{t_{i+p} - t_i} N_{i,p-1}(t) - \frac{p}{t_{i+p+1} - t_{i+1}} N_{i+1,p-1}(t)
\]

\end{proof}

\end{appendices}

\end{document}