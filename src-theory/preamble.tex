\documentclass[11pt,oneside]{book}

\usepackage[margin=1in]{geometry}
\usepackage{graphicx}
\usepackage{amsmath}
\usepackage[colorlinks,bookmarks,bookmarksnumbered,allcolors=blue]{hyperref}
\usepackage[caption=false,justification=raggedright,singlelinecheck=false]{subfig}
\usepackage{booktabs} % better tables
\usepackage[capitalise]{cleveref}
\usepackage{floatrow}
\floatsetup[table]{capposition=top}
\usepackage{listings}
\usepackage[usenames,dvipsnames]{xcolor} 

\graphicspath{{figures/}}



%%
%% Julia definition (c) 2014 Jubobs
%%
\lstdefinelanguage{Julia}%
{morekeywords={abstract,break,case,catch,const,continue,do,else,elseif,%
		end,export,false,for,function,immutable,import,importall,if,in,%
		macro,module,otherwise,quote,return,switch,true,try,type,typealias,%
		using,while},%
	sensitive=true,%
	alsoother={\$},%
	morecomment=[l]\#,%
	morecomment=[n]{\#=}{=\#},%
	morestring=[s]{"}{"},%
	morestring=[m]{'}{'},%
}[keywords,comments,strings]%

\lstset{%
	% language         = Julia,
	basicstyle       = \ttfamily\footnotesize,
	keywordstyle     = \bfseries\color{blue},
	stringstyle      = \color{magenta},
	commentstyle     = \color{ForestGreen},
	showstringspaces = false,
	breaklines=true,
}